% indentation commands
\newcommand{\nid}{\noindent}
\newcommand{\zeroIndent}{\setlength{\parindent}{0cm}}
\newcommand{\oneHalfIndent}{\setlength{\parindent}{1.25cm}}

% remove abstract name
\renewcommand{\abstractname}{\vspace{-\baselineskip}}

% header command
\newcommand{\header}[1]{
  \thispagestyle{plain}
  \fancyhead[C]{#1}
}

% reference style command
\newcommand{\referenceStyle}{
  \pagestyle{plain}
  \small
  \singlespacing
  \zeroIndent
}

% figure command
\newcommand{\insertFigure}[5]{
  \vspace{0.5cm}
  \begin{figure}[H]
    \centering
    \caption{#1}
    \includegraphics[#2\textwidth]{#3}
    \label{fig:#4}

    \centering \footnotesize Fonte: #5
  \end{figure}
  \vspace{0.5cm}
}

% keywords command
\newcommand{\keywords}[1]{
  \vspace{0.5cm}
  \nid \textbf{Palavras-chave:} #1.
}

% image reference command
\newcommand{\refBegin}{
  \centering
  \footnotesize
}

\newcommand{\refEnd}{
  \normalsize
  \justify
}

% quote command
\newcommand{\qtBegin}{
  \small
  \singlespacing
  \zeroIndent
}

\newcommand{\qtEnd}{
  \onehalfspacing
  \normalsize
  \vspace{0.5cm}
  \oneHalfIndent
}

% abstract title command
\newcommand{\abstractTitle}[1]{
  \thispagestyle{plain}
  \zeroIndent
  \centering
  {\fontfamily{lmr}\textsc{\textbf{#1}}}
  \justify
}

% page jump command
\newcommand{\jumpTwoPages}{
  \newpage
  \thispagestyle{plain}
  \phantom{blabla}
  \newpage
}

% thesis settings command
\newcommand{\thesisSettings}{
    % continuous list (tables and images)
    \counterwithin{figure}{section}
    \counterwithin{table}{section}

    % headers
    \pagestyle{fancy}
    \setlength\headheight{15pt}
    \fancyhead[L]{}
    \fancyhead[R]{}

    % disable page numbering
    \pagenumbering{gobble}

    % add dots to table of contents
    \RedeclareSectionCommand[
      toclinefill=\TOCLineLeaderFill
    ]{chapter}
}

% university commands
\newcommand{\ufabc}{Universidade Federal do ABC}
\newcommand{\cecs}{Centro de Engenharia, Modelagem e Ciências Sociais Aplicadas}
\newcommand{\economics}{Bacharelado em Ciências Econômicas}
\newcommand{\thesisTitle}{Título da tese}
\newcommand{\authorName}{Pedro Henrique Rocha Mendes}
\newcommand{\city}{São Bernardo do Campo - SP}
\newcommand{\cityDate}{\city, 1 de outubro de 2022}
\newcommand{\advisor}{Orientador}
\newcommand{\courseDetails}{Trabalho de Conclusão de Curso (TCC) para obtenção do grau de Bacharel em Ciências Econômicas pela Universidade Federal do ABC.}
